% This LaTeX was auto-generated from MATLAB code.
% To make changes, update the MATLAB code and export to LaTeX again.

\documentclass{article}

\usepackage[utf8]{inputenc}
\usepackage[T1]{fontenc}
\usepackage{lmodern}
\usepackage{graphicx}
\usepackage{color}
\usepackage{listings}
\usepackage{hyperref}
\usepackage{amsmath}
\usepackage{amsfonts}
\usepackage{epstopdf}
\usepackage{matlab}

\sloppy
\epstopdfsetup{outdir=./}
\graphicspath{ {./Homework_images/} }

\begin{document}

\begin{par}
\begin{flushleft}
1.说出以下三条指令产生的结果各属于哪种数据类型,是“双精度”对象,还是“符号”对象?
\end{flushleft}
\end{par}

\begin{par}
\begin{flushleft}
\texttt{3/7+0.1, sym(3/7+0.1), vpa(sym(3/7+0.1))}
\end{flushleft}
\end{par}

\begin{par}
\begin{flushleft}
2. 在不加专门指定的情况下,以下符号表达式中的哪一个变量被认为是独立自由变量。
\end{flushleft}
\end{par}

\begin{par}
\begin{flushleft}
\texttt{sym('sin(w*t)') , sym('a*exp(-X)' ) , sym('z*exp(j*theta)')}
\end{flushleft}
\end{par}

\begin{par}
\begin{flushleft}
3. 求符号矩阵$A=\left\lbrack \begin{array}{ccc}
a_{11}  & a_{12}  & a_{13} \\
a_{21}  & a_{22}  & a_{23} \\
a_{31}  & a_{32}  & a_{33} 
\end{array}\right\rbrack$的行列式值和逆,所得结果应采用“子表达式置换”简洁化。
\end{flushleft}
\end{par}

\begin{par}
\begin{flushleft}
4. 对函数$f\left(k\right)=\left\lbrace \begin{array}{cc}
a^k  & k\ge 0\\
0 & k<0
\end{array}\right.$,当$a$为正实数时,求$\sum_{k=0}^{\infty \text{ }} f\left(k\right)z^{-k}$。(实际上,这就是根据定义求Z变换问题。)
\end{flushleft}
\end{par}

\begin{par}
\begin{flushleft}
5. 对于$x>0$,求$\sum_{k=0}^{\infty \text{ }} \frac{\text{ }2}{2k+1}{\left(\frac{x-1}{x+1}\right)}^{2k+1}$。(提示:理论结果为$\mathrm{ln}\text{ }x$)
\end{flushleft}
\end{par}

\begin{par}
\begin{flushleft}
6. (1)通过符号计算求$y\left(t\right)=\mathrm{sin}\text{ }t$的导数$\frac{d\text{ }y}{d\text{ }t}$。(2)然后根据此结果,求$\left.\frac{dy}{dt}\right\vert_{t=0^-}$和$\left.\frac{dy}{dt}\right\vert_{t=\frac{\pi}{2}}$。
\end{flushleft}
\end{par}

\begin{par}
\begin{flushleft}
7. 求出$\int_{-5\pi \text{ }}^{1\ldotp 7\pi \text{ }} e^{-|x|} |\mathrm{sin}\text{ }x|\text{ }d\text{ }x$的具有64位有效数字的积分值。
\end{flushleft}
\end{par}

\begin{par}
\begin{flushleft}
8. 计算二重积分$\int_1^2 \int_1^{x^2 } \left(x^2 +y^2 \right)d\text{ }y\text{ }d\text{ }x$。
\end{flushleft}
\end{par}

\end{document}
